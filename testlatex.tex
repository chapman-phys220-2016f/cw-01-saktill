\documentclass[aps,pra,notitlepage,amsmath,amssymb,letterpaper,12pt]{revtex4-1}
\usepackage{amsthm}
\usepackage{graphicx}
 
%  Helpful commands to set up problem environments easily
\newenvironment{problem}[2][Problem]{\begin{trivlist}
\item[\hskip \labelsep {\bfseries #1}\hskip \labelsep {\bfseries #2.}]}{\end{trivlist}}
\newenvironment{solution}{\begin{proof}[Solution]}{\end{proof}}
 
% --------------------------------------------------------------
%                   Document Begins Here
% --------------------------------------------------------------
 
\begin{document}
 
\title{Classwork 1}
\author{Saktill}
\affiliation{PHYS 220 -- Scientific Computing I}
\date{\today}

\maketitle

%
\begin{problem}{x.yz} 
Explain what the definition $f'(x)$ of a function $f(x)$ means.
\end{problem}
 
\begin{solution}
The definition of $f'(x)$ is given as
\[
\begin{split}
f'(x) =  lim_{h\to 0} frac{f(x+h) - f(x)}{h}
\end{split}
\]
In plain words this means that the rate of change of a function is given by taking a small dx along the function, and taking the limit as that small dx goes to zero of the difference between $f(x+h)$ and $f(x)$ and dividing by the small change. This limit gives the instantaneous rate of change.




\end{solution}



\end{document}




% Repeat as needed
 
 
\end{document}
